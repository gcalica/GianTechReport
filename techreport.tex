\documentclass[english]{proposalnsf}
\usepackage{graphicx}
\usepackage{url}
\usepackage[square,numbers]{natbib}
\usepackage[nottoc,numbib]{tocbibind}

\title{Improved understanding of student interests in Computer Science through Page Tracking Analysis}
\author{Gian Calica \\Collaborative Software Development Laboratory \\ Information and Computer Sciences \\ University of Hawaii}

\begin{document}
  \maketitle
  \tableofcontents
  \newpage

  \section{Introduction}
  \label{sec:introduction}
  The field of computer science has rapidly expanded in such a short amount of time over the last few decades and is
  continuing to expand at a growing rate.
  As it continues to expand, so must also the technologies that support it.
  Every month there are new technologies being released, whether it's a programming language, a framework, or a new
  library encapsulating domain knowledge or an algorithmic technique for a language, all of which claim to have an
  advantage over a previous technology.

  As the field continues to expand, it is important for the Information and Computer Sciences (ICS) Department in
  University of Hawaii at Manoa (UHM) to be able to respond and keep up with these fast-changing developments.
  RadGrad is one way the ICS Department can tell which courses to include in its curriculum. RadGrad can track how
  many students plan to take certain courses and in which semesters, and this statistic can be used by the students as
  a way to tell the ICS Department which type of courses they are interested in.

  RadGrad also has a way to track how many students are interested in a particular area of Computer Science through
  interests. However, many of these interests have no course available in the ICS Department to teach students about
  those interests. This proposal proposes a mechanism called ``Page Tracking Analysis'' which can discover existing
  or emergent demand for topic areas by computer science students.

  \section{Research Design}
  \label{sec:research-design}

  In the context of the RadGrad system, we will define a student's interest in a particular topic with two definitions: 1) if the student has visited the corresponding page for that topic in RadGrad and 2) if the student has added that topic to their favorites.

  With those definitions in mind, this proposal will implement a capability called the Page Tracking Analysis. It is
  implemented like a scoreboard because the RadGrad2 system already has similar pages called scoreboards that function
  similarly but for different data.

  In this dedicated page, there will be a section for each topic category: Interests, Career Goals, Interests, Courses,
  and Opportunities. Each section will list the names for that area and beside it would be the page visits for each
  corresponding name. What will make this page powerful is that we can filter each of the names to show us which are
  the Most Viewed or Least Viewed by page visits and the standard Alphabetical filter.

  There is another filter that we can add: Trending. In media streaming websites, a common filter is ``Trending'' or
  ``Most Popular''. And this filter can be further narrowed down to a date range such as ``Today,''
  ``Past Week,'' ``Past Month'', ``Past Year.'' This filter will show the trending or most popular areas within a date
  range. When it comes to the date range, the faculty can choose a specified date range rather than a user's
  perspective (``Today,'' ``Past Week,'' ``Past Month'', ``Past Year.'').

  With these statistics being tracked, we can now analyze and see if there is a demand for a particular topic that the
  faculty does not currently know through these page visits. One question that could be explored would be, ``Why did
  the interest Data Science and career goal Data Scientist suddenly spike up in page visits in this certain
  date range?'' Then when we investigate this cause, we might find that a course related to Data Science was actually
  offered in that semester of that date range. From this research investigation, we might conclude that our RadGrad
  team should promote and look for opportunities related to Data Science in that particular date range.

  Just like the purpose of collecting ``Students Participating'' data for a Course or Opportunity, the ICS faculty and
  department can also use this dataset to see which areas they should emphasize and focus on teaching the students.

  As students also explore the field of computer science beyond what their classes teach, analyzing which fields in
  computer science they are most interested in allows us to also have a glimpse of what the field of computer science
  is heading towards. This can better prepare the ICS Department to adjust the courses and curriculum to meet with the
  standards of the constantly expanding computer science industry.

  \section{Implementation}
  \label{sec:implementation}
  Although we define a student visiting the page of a particular topic as the student being interested in that topic,
  we do not want to count each page visit counting towards their interest. There are certain circumstances such as
  accidentally clicking the page even though they were not interested in that topic that might skew the data. I need to
  figure out an algorithm or system to determine this. Some factors to consider are time between a student visiting the
  page for a topic versus a student leaving the page or time between a student visiting the page for a topic versus a
  student adding that topic to their favorites.

  Since the data of page visits is tracked and can only be queried from the UserInteractions collection, this raises a
  few obstacles in the implementation. The first obstacle is that querying from UserInteractions is only permitted for
  Administrator roles. Therefore, any other roles such as Faculty or Advisor will not even get the data when they load
  up the page. The second obstacle is that UserInteractions is the biggest collection in the RadGrad system. Besides
  page visits, UserInteractions also track other statistics such as student interactions with their profile. Therefore,
  querying from this collection directly will lead to time complexity issues. To overcome these obstacles, a separate
  collection must be made to specifically track the page visits for each topic area. Querying from this collection will
  be available to Administrators, Faculty, and Advisors. In the UserInteractions collection, each student interaction
  (such as a page visit) is one document object in that collection. This is the main reason why the UserInteraction
  collection is massive. In this new collection, a document will contain the value of the sum of the page visits for
  all the topic areas.

  \section{Data Analysis}
  \label{sec:data-analysis}
  This section is about how to represent the data.

  \section{Evaluation}
  \label{sec:evaluation}
  This section is about discussing how to evaluate the effectiveness of this scoreboard once it has been implemented.

  \appendix

\end{document}

